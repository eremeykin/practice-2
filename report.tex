\documentclass[12pt,twoside,a4paper,tikz,border=5]{refart} 
% add russian language
\usepackage[utf8]{inputenc}
\usepackage[T1]{fontenc}
\usepackage[russian]{babel}

\usepackage{indentfirst}
\setlength{\parindent}{1cm}

% setup titles 
\usepackage{titlesec}
	\titleformat{\section}
		{\normalfont\fontsize{16}{0}\bfseries}{\thesection}{1em}{}
	\titleformat{\subsection}
		{\normalfont\fontsize{14}{0}\bfseries}{\thesubsection}{1em}{}
	\titleformat{\subsubsection}
		{\bf\fontsize{14}{0}}{\thesubsubsection}{1em}{}

	\titlespacing*{\section}
		{0pt}{5ex}{3ex} % left before after
	\titlespacing*{\subsection}
		{0pt}{3ex}{1ex}
	\titlespacing*{\subsubsection}
		{0pt}{3ex}{1ex}
	%start each section on new page
	\newcommand{\sectionbreak}{\clearpage}


% set up better resolution
\pdfpkmode{dpdfezzz}
\pdfpkresolution=8000

% set fraction of text
\settextfraction{0.88}

%colorbox for menu
\usepackage[most]{tcolorbox}
%\newtcbox{\menubox}{nobeforeafter,colframe=mycolor,colback=mycolor!10!white,boxrule=0.5pt,arc=4pt,
%	boxsep=0pt,left=6pt,right=6pt,top=6pt,bottom=6pt,tcbox raise base}

\definecolor{myblue}{rgb}{0.000, 1.000, 1.000}
\definecolor{dodgerblue}{rgb}{0.117,0.564,1.000}

\newtcbox{\menubox}{enhanced,nobeforeafter,tcbox raise base,boxrule=0.4pt,top=0mm,bottom=0mm,
	right=1mm,left=1mm,arc=2pt,boxsep=2pt,before upper={\vphantom{dlg}},
	colframe=blue!50!black,coltext=black,colback=myblue,
}

\robustify{\menubox}

\newtcbox{\filebox}{enhanced,nobeforeafter,tcbox raise base,boxrule=0.4pt,top=0mm,bottom=0mm,
	right=0mm,left=4mm,arc=1pt,boxsep=2pt,before upper={\vphantom{dlg}},
	colframe=green!50!black,coltext=green!25!black,colback=myblue,
	overlay={\begin{tcbclipinterior}\fill[dodgerblue] (frame.south west)
			rectangle node[text=white,font=\sffamily\bfseries\tiny,rotate=90] {FILE} ([xshift=4mm]frame.north west);\end{tcbclipinterior}}}

\robustify{\filebox}

\newtcbox{\dirbox}{enhanced,nobeforeafter,tcbox raise base,boxrule=0.4pt,top=0mm,bottom=0mm,
	right=0mm,left=4mm,arc=1pt,boxsep=2pt,before upper={\vphantom{dlg}},
	colframe=green!50!black,coltext=green!25!black,colback=myblue,
	overlay={\begin{tcbclipinterior}\fill[dodgerblue] (frame.south west)
			rectangle node[text=white,font=\sffamily\bfseries\tiny,rotate=90] {DIR} ([xshift=4mm]frame.north west);\end{tcbclipinterior}}}

\robustify{\dirbox}

% for figures
\usepackage{subcaption}
\DeclareCaptionLabelFormat{gostfigure}{Рисунок #2}
\DeclareCaptionLabelFormat{slide}{Слайд №#2}
\DeclareCaptionLabelFormat{gosttable}{Таблица #2}
\DeclareCaptionLabelSeparator{gost}{~---~}

\captionsetup{labelsep=gost}
\captionsetup[figure]{labelformat=gostfigure}
\captionsetup[table]{labelformat=gosttable}
\captionsetup{figurewithin=none, tablewithin=none}
\renewcommand{\thesubfigure}{\asbuk{subfigure}}

%for crop
\usepackage{graphicx}

% for tables and image inside
\usepackage{array}
\usepackage{tabularx,booktabs} 
% biblio style
\bibliographystyle{utf8gost705u} 

% for margins
\setlength\oddsidemargin{-0in}
\setlength\evensidemargin{-0in}
\setlength\textwidth{165mm}

% for web links
\usepackage{hyperref}
\hypersetup{
	colorlinks=true,
	linkcolor=blue,
	urlcolor=blue,
}

\urlstyle{same}

% for header on every page
\usepackage{fancyhdr}
\usepackage{lipsum}% just to generate text for the example

\pagestyle{fancy}
\fancyhf{}
\fancyhead[L]{\rightmark}
\fancyhead[R]{\thepage}
\renewcommand{\headrulewidth}{0.4pt}
%\renewcommand{\footrulewidth}{0.4pt}% default is 0pt

% Tables and pictures captions align
\usepackage[]{caption}
\captionsetup[table]{justification=raggedleft} 
\captionsetup[figure]{justification=centering,labelsep=endash} 

% for enumerate style
\renewcommand\labelenumii{\theenumi.\arabic{enumii}.}

% for pagebreak in tabularx
\usepackage{ltablex}

\newcounter{titem}[table]
\newenvironment{titem}[2][]{\refstepcounter{titem}\par\medskip
	\noindent \textit{
		\begin{flushleft}
			\begin{hyphenrules}{nohyphenation}
				\thetitem.~#1~#2\newline
			\end{hyphenrules}
		\end{flushleft}
	}}{\medskip}

\begin{document}
	\begin{titlepage}
		\centering
		{\scshape\small Национальный исследовательский университет \\
			``Высшая школа экономики'' \par}
		\vspace{2cm}
		{\Huge\bfseries ОТЧЁТ ПО ПРАКТИКЕ\par}
		\vspace{2cm}
		{\scshape\Large Исследование возможного сокращения перебора при выборе параметров $ p, \beta $ \\
			для алгоритма $ A-Ward_{p\beta}$ \par}
		\vspace{1.5cm}
				
		\vfill
		\begin{center}
			\begin{tabular}{  p{7cm}  p{4cm} p{5cm}  } 
				& & \textbf{Студент:}\\ 
				& & Еремейкин П.А. \\ 
				& & группа \\
				& & мНоД16\_ТМСС\\
				& & \\
				& & \textbf{Руководитель: }\\
				& & профессор\\
				& & Миркин Б.Г.\\
			\end{tabular}
		\end{center}
		\vfill
		{Москва \the\year\par}
	\end{titlepage}


	\tableofcontents 
	\newpage
	\section{Основные положения}
	
		Исследование выполняется в рамках развития пакета программ СИК (Система Интеллектуальной Кластеризации), который был разработан в ходе курсового проекта ``Алгоритмы интеллектуализации метода k-средних''. Этот пакет предназначен для применения современных интеллектуальных методов при решении задач кластеризации. 
		
		В состав пакета входят методы иерархического кластер-анализа: метод аномальных кластеров, алгоритмы $ Ward $, $ A-Ward $, $ A-Ward_{p\beta} $ а также дивизивные методы. 
		
		С технической точки зрения СИК представляет собой набор Python модулей, объединённых в единую программу при помощи графического пользовательского интерфейса.  
		
		В рамках данной практики рассматривается проблема выбора параметров для алгоритма $ A-Ward_{p\beta} $. Этот алгоритм представляет собой модифицированную версию иерархического алгоритма $ A-Ward $ и вводит два параметра: $ p $ и $ \beta $. Оптимальные значения параметров зависят от конкретной задачи и данных, к которым применяется алгоритм. На настоящее время не существует рекомендаций по эффективному выбору этих параметров, а единственный обоснованный метод --- перебор всех возможных значений с последующей оценкой результата для каждой пары $ (p_i,\beta_i) $  по эмпирической характеристике. Такой подход требует большого времени вычисления, что во многих случаях делает его неприменимым на практике.
		
		Для решения задачи выбора параметров $ p$, $\beta $ в условиях ограниченного времени была выдвинута гипотеза о возможном сокращении перебора. Согласно этой гипотезе, результаты выбора оптимальных значений по всем доступным объектам и по сокращённой выборке из этих объектов различаются не существенно. Цель данной работы состоит в экспериментальной проверке приведённой гипотезы, оценке различных стратегий формирования сокращённой выборки, их характеристик относительно качества результата и затрачиваемого времени.
	
	\section{Алгоритм $ A-Ward_{p\beta} $}
		\subsection{Постановка задачи. Кластеризация}
			Алгоритм $ A-Ward_{p\beta} $ предназначен для решения задачи кластеризации, то есть выделения из таблиц наблюдения множеств (кластеров) таким образом, чтобы сходные объекты попадали в один и тот же кластер, а несходные --- в разные кластеры \cite{data-science}. 
	
	
	\section{Методика эксперимента}
	\section{Экспериментальное обеспечение}
	\section{Результаты}
	\section{Выводы}
	
	% Список литературы
	\newpage
	\nocite{amorim} \nocite{ward} \nocite{boley} \nocite{tasoulis} \nocite{kovaleva} \nocite{mirkin}
	\addcontentsline{toc}{section}{Список литературы}
	\bibliography{bibliography}
	
	

\end{document}